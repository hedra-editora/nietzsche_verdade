
\vspace*{7cm}

\noindent\textbf{Friedrich Nietzsche} (Röcken, 1844--Weimar, 1900), filósofo 
e filólogo alemão, foi crítico mordaz da cultura ocidental 
e um dos pensadores mais influentes da modernidade. Descendente de pastores 
protestantes, opta no entanto por seguir carreira acadêmica. 
Aos 25 anos, torna-se professor de letras clássicas na Universidade 
da Basileia, onde se aproxima do compositor Richard Wagner. Serve 
como enfermeiro voluntário na guerra franco-prussiana, mas contrai 
difteria, a qual prejudica a sua saúde definitivamente. Retorna a 
Basileia e passa a frequentar mais a casa de Wagner. Em 
1879, devido a constantes recaídas, deixa a universidade e passa a 
receber uma renda anual. A partir daí assume uma vida errante, 
dedicando-se exclusivamente à reflexão e à redação de suas obras, 
dentre as quais se destacam: \textit{O nascimento da tragédia} (1872), 
\textit{Assim falava Zaratustra} (1883--1885), \textit{Para além do bem e mal} (1886), 
\textit{A genealogia da moral} (1887) e \textit{O anticristo} (1895). Em 1889, 
apresenta os primeiros sintomas de problemas mentais, provavelmente 
decorrentes de sífilis. Falece em 1900.

\noindent\textbf{Sobre verdade e mentira no sentido extramoral} (\textit{Über Wahrheit 
und Lüge im außermoralischen Sinn}) é um opúsculo que investiga 
o alcance efetivo da linguagem, sobre a qual se assenta todo o 
conhecimento da civilização ocidental. Para Nietzsche, a confiança do 
homem moderno no poder das palavras se funda no esquecimento de que 
algo que era evidente quando as criou: elas são apenas uma metáfora 
para as coisas e jamais poderiam encarnar o seu significado. 
Constata-se, portanto, um desacerto entre o conhecimento intuitivo 
e as abstrações conceituais. Esta obra foi ditada pelo autor a um amigo 
no verão de 1873, mas só foi publicada após a sua morte. A presente 
edição contém ainda uma seleta oportuna de \textit{Fragmentos póstumos}.

\noindent\textbf{Fernando de Moraes Barros} é doutor em filosofia pela Universidade 
de São Paulo (\textsc{usp}) e professor da Universidade Estadual de Santa Cruz 
(\textsc{uesc}). É autor de \textit{A maldição transvalorada -- o problema da civilização 
em} O anticristo \textit{de Nietzsche} (Discurso/Unijuí, 2002) 
e \textit{O pensamento musical de Nietzsche} (Perspectiva, 2007).
