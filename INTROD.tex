\chapter[Introdução, por Fernando de Moraes Barros]{Introdução}
\hedramarkboth{Introdução}{Fernando de Moraes Barros}


\noindent\textsc{De todos} os textos de Nietzsche, \textit{Sobre verdade e mentira no sentido
extra"-moral} é decerto um dos mais singulares e pregnantes. Ditado ao
colega K.~von Gersdorff em junho de 1873, o escrito é fruto não apenas
de uma refinada espiritualidade, mas também de um importante
redimensionamento teórico"-especulativo. À diferença de ponderações
anteriores, nele o filósofo alemão --- à época, professor na Universidade
da Basileia --- já não toma a palavra a fim de caracterizar o despertar
da tragédia ática. Tomado por novos planos e interesses,
abandona"-se agora a novas auto{}-satisfações. Pensador livre e laico,
debruça"-se sobre as assim chamadas ciências da natureza,
comprazendo"-se na leitura de textos tais como, por exemplo,
\textit{Philosophiae naturalis theoria} de R.~J.~Boscovich. Luz a
eliminar preconceitos e intolerâncias, o espírito contido nos métodos científicos
talvez ajude a desanuviar as sombras metafísicas que se acumulam em
torno do conhecimento. Mais até. No momento em que aprende a questionar
a si mesma, a verdade talvez termine por revelar alguma não"-verdade à
sua base, prestando um testemunho inteiramente inesperado sobre si
própria. É precisamente essa suspeita que vigora em \textit{Sobre
verdade e mentira no sentido extra"-moral}.

Movida pela crença de que a forma fundamental do pensamento é a mesma
de suas manifestações por palavras, desde cedo, a filosofia não hesitou em
identificar discurso e realidade. Concebendo o pensar como uma
inequívoca atividade de simbolização enunciativa, ela parece ter sempre
dado atenção especial à dimensão apofântica da linguagem, tomando
enunciados verbais por verdadeiros ou falsos, em função de descreverem
corretamente ou não o mundo. O que ocorreria, porém, se a verdade dos
enunciados não passasse de um tipo de engano sem o qual o homem não
poderia sobreviver? E se a condição da verdade fosse a mesma da
mentira? Revelar"-se{}-ia, então, o atávico caráter dissimulador do
intelecto humano e, com ele, a suspeita de que entre o “refletir” e o
“dizer” não vigora nenhuma identidade estrutural. É justamente a essa
conclusão que Nietzsche espera conduzir"-nos.

O caminho encontrado pelo filósofo alemão para abordar a questão
não se inscreve num registro tradicional.
Negando"-se a
separar o homem da natureza, sua abordagem procura mostrar que foi para
satisfazer às injunções imediatas de sobrevivência que os seres humanos
forjaram e aprimoraram o conhecimento. Servindo ao desejo de
conservação imposto pela gregariedade, o intelecto priorizaria noções
aptas a assegurar a vida em conjunto e, pelo mesmo trilho, 
seria obrigado a produzir falsificações. Nesse sentido, 
lê"-se: 
\begin{hedraquote}
Como um meio para a
conservação do indivíduo, o intelecto desenrola suas principais forças
na dissimulação; pois esta constitui o meio pelo qual os indivíduos
mais fracos, menos vigorosos, conservam"-se, como aqueles aos quais é
denegado empreender uma luta pela existência com chifres e presas
afiadas. No homem, essa arte da dissimulação atinge seu
cume.\footnote{ Friedrich Nietzsche, \textit{Sämtliche Werke.
Kritische Studienausgabe}, Giorgio Colli e Mazzino Montinari,
Berlim / Nova York, Walter de Gruyter, 1999, p.~876.}
\end{hedraquote}
Por ser criada sob a pressão da necessidade de comunicação e sociabilidade, a
consciência de si não faria parte, em rigor, da existência do indivíduo
enquanto tal, mas de sua interação com o meio e aqueles que o rodeiam,
referindo"-se àquilo que nele há de comum e trivial. Admitir isso, porém,
implica aceitar que os recursos de que o pensamento se serve para
ganhar forma e conteúdo são pré"-formados pela coletividade, de sorte
que estaríamos fadados a exprimir nossos raciocínios sempre com as
palavras que se acham à disposição de todos. A esse respeito, Nietzsche
escreve: 
\begin{hedraquote}
Quando justamente a mesma imagem foi gerada milhões de vezes
e foi herdada por muitas gerações de homens [\ldots{}] então ela termina por
adquirir, ao fim e ao cabo, o mesmo significado para o homem, como se
fosse a imagem exclusivamente necessária [\dots] assim como um sonho que
se repete eternamente seria, sem dúvida, sentido e julgado como
efetividade. \footnote{ Id.~ibid., p.~884.} 
\end{hedraquote}
Reincidentes,
as experiências em comum com o outro terminariam por se sobrepor
àquelas que ocorrem com menor frequência no seio da coletividade. Sem
ter acesso, em princípio, a outras palavras, o indivíduo tampouco
teria facilidade para liberar aquelas de que dispõe para outras
aplicações. Resignado a tal inacessibilidade, ele é livre somente
para falar e pensar como os outros.

Com efeito, dizer que são as palavras comumente partilhadas que
possibilitam a conscientização do próprio sentir e pensar impele,
ao menos, a uma relevante consequência: a de que aquilo que o homem
sente e pensa a respeito de si mesmo já se encontra
condicionado pelas mais elementares estruturas da linguagem. 
Para Nietzsche, todavia, as palavras nos iludem quando as
tomamos à risca e deixamos de perceber, por meio delas, acontecimentos
que elas mesmas não podem assimilar. A seu ver, o pensamento 
tornado consciente seria apenas um produto acessório do intrincado
processo psíquico que o atravessa e constitui. 
Quando é vertida em palavras e
signos de comunicação, a atividade reflexiva já se acharia circunscrita
à esfera da calculabilidade, e estaria inserida em esquemas longamente
consolidados de simplificação e abstração, com vistas ao nivelamento
identificador do fluxo polimorfo do vir"-a{}-ser e da natureza.

Visto como um epifenômeno de nossas funções orgânicas fundamentais, o
pensamento adquire, então, um sentido ligado a um universo
infra"-consciente bem mais recuado, que engloba processos vitais cujo
sentido último sempre nos escaparia. Ao dispensar uma subjetividade que
os estabelecesse e determinasse, tais processos reguladores assumem
um significado associado a recônditas operações do corpo, não mais de
uma consciência pensante detentora de suas representações, que, de
resto, não passaria de um mero vetor auxiliar ou instrumento diretivo.
A esse respeito, lê"-se ainda:

\begin{hedraquote}
O que sabe o homem, de fato, sobre si mesmo! [\ldots{}] Não se lhe emudece
a natureza acerca de todas as outras coisas, até mesmo acerca de seu
corpo, para bani"-lo e trancafiá{}-lo numa consciência orgulhosa e
enganadora, ao largo dos movimentos intestinais, do veloz fluxo das
correntes sanguíneas e das complexas vibrações das fibras! Ela jogou
fora a chave: e coitada da desastrosa curiosidade que, através de uma
fissura, fosse capaz de sair uma vez sequer da câmara da consciência e
olhar para baixo, pressentindo que, na indiferença de seu não"-saber,
o homem repousa sobre o impiedoso, o voraz, o
insaciável.\footnote{ Id.~ibid., p.~877.}
\end{hedraquote}

Para chegar a compreender melhor como a linguagem exerce seu efeito
dissimulador sobre aquilo que o homem sente e pensa sobre si mesmo,
impõe"-se saber o que são as próprias palavras. Questão essa à qual se
responde: 
\begin{hedraquote}
De antemão, um estímulo nervoso transposto em uma imagem!
Primeira metáfora. A imagem, por seu turno, remodelada num som! Segunda
metáfora.\footnote{ Id.~ibid., p.~879.} 
\end{hedraquote}
Duplo, o processo de formação
da palavra comportaria a seguinte transposição: uma excitação nervosa
convertida numa imagem mental e, em seguida, a transposição de tal
imagem num som articulado. Heteróclita, a passagem operaria, em rigor,
com elementos que pertencem a esferas disjuntivas, de sorte
que uma correspondência biunívoca entre coisas e palavras só poderia
ser obtida pela negação da distância que separa a sensação
experimentada pelo indivíduo e o som por ele emitido. Ao acreditar que
cada palavra pronunciada designa algo bem definido e acertado acerca do
mundo exterior, ele mal pressente que se trata, aqui, de domínios desiguais.

Mas, precisamente por que a palavra foi criada para exprimir uma
sensação subjetiva, ela só pode referir"-se às relações entre as
coisas e nós mesmos, nunca às próprias coisas: 
\begin{hedraquote}
Acreditamos saber algo
acerca das próprias coisas, quando falamos de árvores, cores, neve e
flores, mas, com isso, nada possuímos senão metáforas das coisas, que
não correspondem, em absoluto, às essencialidades originais.\footnote{
Id.~ibid., p.~879.}
\end{hedraquote}
Todavia, desejoso de encontrar correlatos para as
palavras que veicula, o indivíduo abrevia aquilo que se lhe apresenta
conforme seus interesses, optando, de modo unilateral, ora por este,
ora por aquele aspecto da efetividade. Niveladora, a linguagem da qual
ele se serve depende da igualação do não"-igual para adquirir
autovaloração, o que se tornaria patente, por exemplo, na própria
constituição dos conceitos: 
\begin{hedraquote}
Tão certo como uma folha
nunca é totalmente igual a uma outra, é certo ainda que o conceito de
folha é formado por meio de uma arbitrária abstração dessas diferenças
individuais, por um esquecer"-se 
do diferenciável.\footnote{ Id.~ibid.~, p.~880.}
\end{hedraquote}

Tomada num sentido unívoco e inabalável, no sentido que lhe foi dado em
todas as épocas, a palavra mesma passa a ser vista como existindo
\textit{ad aeternum}. Instituída num tempo adâmico, o falante talvez
até acreditasse que ela adquire realidade num mundo
supra"-sensível. Contrariando esse estado de coisas, o filósofo alemão
empreende a pergunta pela produção mesma do signo linguístico e, ao
fazê"-lo, termina por colocar a questão acerca das circunstâncias de
seu aparecimento. Com isso, pretende conduzir"-nos à ideia de que, na
linguagem, o que vigora não é a imobilidade de sentido e tampouco uma
estrutura invariável dotada de significação idêntica, mas 
\begin{hedraquote}
Um exército
móvel de metáforas, metonímias, antropomorfismos, numa palavra, uma
soma de relações humanas que foram realçadas poética e
retoricamente.\footnote{ Id.~ibid., p.~880.}
\end{hedraquote}

Porque passa ao largo dessa profusão de formas e figuras, a compressão
essencialista da linguagem revela"-se, desde logo, uma fonte
inesgotável de auto"-enganos. Tomando acidentes por substâncias e
relações por essências, ela transpõe e inverte as categorias que ela mesma
se dedica a engendrar; substituindo coisas por significados, faz crer
que as designações e as coisas se recobrem e, com isso, ilude quem nela
procura fiar"-se;\footnote{ “O conceito ‘lápis’” --- escreve Nietzsche ---
“é trocado pela ‘coisa’ lápis.” (Id.~Fragmento póstumo do verão de
1872, n° 19 [242]; em \textit{Sämtliche Werke. Kritische
Studienausgabe}, Giorgio Colli e Mazzino Montinari, Berlim / Nova
York, Walter de Gruyter, 1999, vol.~7, p.~495).} condicionando o homem
ao hábito gramatical de interpretar a realidade vendo nela apenas
sujeitos e predicados, incita"-o a postular a existência de um autor
por detrás de toda ação; enquadrando aquilo que os seres humanos pensam
e falam nos padrões da causalidade, tal concepção os impele, em
suma, a negar o caráter processual da existência.

A exigência analítica de um modo de expressão perfeitamente
adequado e objetivo, qual um decalque transparente da esfera que
designa a efetividade, só poderia ganhar relevo, no fundo, pela falta de
cautela crítica. Daí a oportunidade descerrada por Nietzsche de
combater a ideia de que se possa obter, por meio das palavras, um
acesso ao núcleo indivisível e inquestionável do existir. 
A seu ver, a verdade que as
palavras nos colocariam em mãos seria de ordem tautológica. Através
delas, o homem apenas reencontraria aquilo que ele próprio teria
introduzido nas designações. A fim de esclarecer essa curiosa espécie
de auto"-ofuscamento, o filósofo alemão provê o seguinte exemplo:
\begin{hedraquote}
Quando alguém esconde algo detrás de um arbusto, volta a
procurá"-lo justamente lá onde o escondeu e além de tudo o encontra,
não há muito do que se vangloriar nesse procurar e encontrar [\ldots{}] Se
crio a definição de mamífero e, aí então, após inspecionar um camelo,
declaro: veja, eis um mamífero, com isso, uma verdade decerto é trazida
à plena luz, mas ela possui um valor limitado.\footnote{ Id.~ibid., p.~883.}
\end{hedraquote}

O processo que consiste em definir o conceito de animal mamífero para,
a partir de um animal particular, compor o enunciado “Veja,
eis um mamífero”, teria como consequência a ideia de que o “ser”
mamífero pertenceria essencialmente ao exemplo individual. O que já não
ocorreria no seguinte caso: 
\begin{hedraquote}
Denominamos um homem honesto; perguntamos então: 
por que motivo ele agiu hoje de modo tão
honesto? Nossa resposta costuma ser a seguinte: em função de sua
honestidade.\footnote{ Id.~ibid., p.~880.} 
\end{hedraquote}
A despeito de
figurar como uma propriedade acidental do sujeito da proposição, o
termo “honesto” dá a entender, aqui, que a própria “honestidade”
pertence à essência do sujeito em questão, não só como atributo, mas
como substância, já que foi em virtude de tal termo que a denominação
ganhou sentido, de sorte que a alardeada diferença entre essência e
acidente não seria nada inconcussa, mas inteiramente casual. O que
também revelaria, uma vez mais, a tautologia subjacente à própria
linguagem: o ser do homem honesto estaria, no fundo, no fato de ele ser
honesto.

Assim, se pela definição geral --- animal mamífero, por exemplo -- não se
tem acesso ao “verdadeiro em si”, tampouco caberá às palavras
que se aplicam às propriedades particulares torná"-lo acessível a nós.
Antropomórfica, a oposição entre universal e particular não proviria da
essência das coisas, mas de um abuso: 
\begin{hedraquote}
Nada sabemos, por
certo, a respeito de uma qualidade essencial que se chamasse a
honestidade, mas, antes do mais, de inúmeras ações individualizadas e,
por conseguinte, desiguais, que igualamos por omissão do desigual e
passamos a designar, desta feita, como ações
honestas.\footnote{ Id.~ibid.~, p.~880.}
\end{hedraquote}
Mas se, por aí, o homem não faz senão se enredar na trama de suas
próprias ficções, não lhe seria permitido vislumbrar uma dimensão mais
visceral, através da qual ele pudesse reencontrar não a presença
imediata das coisas em si mesmas, mas aquilo que há de “inexplorado” na
palavra? Na tentativa de responder afirmativamente à pergunta,
Nietzsche espera descobrir e afirmar um modo de representação anterior
à própria palavra articulada, que viria à tona sob a forma de uma
metáfora intuitiva. Acerca desta que poderia ser caracterizada como uma
ancestral remota e fugidia do próprio conceito, ele pondera: 
\begin{hedraquote}
Mesmo o
conceito, ossificado e octogonal como um dado e tão rolante como este,
permanece tão"-somente o \textit{resíduo de uma metáfora}, sendo que a
ilusão da transposição artística de um estímulo nervoso em imagens, se
não é a mãe, é ao menos a avó de todo conceito.\footnote{ Id.~ibid.,
p.~882. }
\end{hedraquote}

Como inequívoca paródia da compreensão do homem acerca da linguagem, a
metáfora intuitiva surge, se não como a mãe, pelo menos enquanto a mãe
da mãe de toda representação conceitual. Mas, evitando investigar a
história de seus “antepassados”, a rede humana de conceitos já não
reconhece as metáforas de origem, como metáforas, e as toma pelas
coisas mesmas. É justamente por proceder dessa maneira que a linguagem
renunciaria à oportunidade de tomar para si outras funções, soterrando
o poder criador e inaudito que traz consigo. A esse propósito, o
filósofo alemão escreve ainda: 
\begin{hedraquote}
A partir dessas intuições nenhum
caminho regular dá acesso à terra dos esquemas fantasmagóricos, [\ldots{}]
o homem emudece quando as vê, ou, então, fala por meio 
de metáforas nitidamente proibidas e combinações conceituais inauditas,
para ao menos corresponder criativamente, mediante o desmantelamento e
a ridicularização das antigas limitações conceituais, à poderosa
intuição atual.\footnote{ Id.~ibid., p.~889.}
\end{hedraquote}

Em vista disso, quem procurasse na linguagem “um novo âmbito para sua
ação'',\footnote{ Id.~ibid., p.~887.} seja por meio de
metáforas proibidas, seja por meio de arranjos conceituais inéditos,
encontraria tal senda, ``em linhas gerais, na
arte.”\footnote{ Id.~ibid., p.~887.} São precisamente as consequências
dessa aceitação que irão impelir Nietzsche, mais tarde, a tentar assegurar
à linguagem não um fundo sonoro supra"-sensível, mas uma musicalidade
atinente à própria palavra. É também por aí que se compreende o motivo
pelo qual a chamada linguagem dos gestos terminará por converter"-se,
como expressão derradeira e paroxística do estilo nietzschiano, na
própria “eloquência tornada música”. Razões bastantes para que a
ponderação contida em \textit{Sobre verdade e mentira no sentido
extra"-moral } possa ser vista como a semente a partir da qual nasce e
cresce a orientação filosófica exigida pelo Nietzsche da maturidade. E
não só. Ao mostrar que a ilusão faz parte dos pressupostos da vida, seu
autor faz ver que nós também, a despeito de nossas portentosas
verdades, mentimos para viver.
